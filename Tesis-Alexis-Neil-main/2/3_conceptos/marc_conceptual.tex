\subsection{Visión por Computadora}
La visión artificial es un campo de la inteligencia artificial (IA) que utiliza el aprendizaje automático y las redes neuronales para enseñar a los ordenadores y sistemas a obtener información significativa de imágenes digitales, vídeos y otros elementos visuales, y a hacer recomendaciones o tomar medidas cuando ven defectos o problemas (\cite{IBM2023vision}).

\subsection{Sistemas de Recomendación}
Los sistemas de recomendación (RS) son una de las soluciones que ayudan a los usuarios a gestionar la sobrecarga de opciones. Las RS son herramientas importantes que permiten a las personas tomar decisiones alineadas con sus necesidades y preferencias (\cite{Amin2020}).

\subsection{Datos No Estructurados}
Los datos no estructurados se definen comúnmente como datos que no están disponibles en formatos estructurados predefinidos, como los formatos tabulares. En la literatura, los datos digitales no estructurados a menudo se denominan indistintamente "big data", "datos digitales", "datos textuales no estructurados" y se describen como "de alta dimensión", "gran escala", "ricos", "multivariados" o "sin procesar" (\cite{sarmiento2023challenges}).


\subsection{YOLO}
Segun (\cite{Redmon2016YOLO}) You Only Look Once (YOLO), es un enfoque innovador que trata sobre examinar las imagenes para detectar objetos y posiciones. Además, en el paradigma YOLO, emplea una sola red neuronal convolucional para predecir cuadros delimitadores y las probabilidades de clase para una imagen completa (\cite{Hussain2023YOLO}).


\subsection{Inteligencia Artificial Aplicada a Retail}
La inteligencia artificial (IA) se refiere a la simulación de procesos de inteligencia humana por parte de máquinas, ejecutadas principalmente a través de sistemas informáticos. Además, segun (\cite{Cocco2022}) La inteligencia artificial (IA) se refiere al desarrollo de sistemas informáticos capaces de realizar tareas que normalmente requieren inteligencia humana, como el aprendizaje, el razonamiento y la toma de decisiones


\subsection{HOG (Histogram of Oriented Gradients)}
Histogram of oriented grafients es HOG es un tipo de “descriptor de características”. El objetivo de un descriptor de características es generalizar el objeto de tal forma que el mismo objeto. (\citet{pardoHOG}). Los descriptores HOG son descriptores utilizados para la detección de objetos, los cuales están basados en histogramas de gradientes orientados (\citep{avilaHOG}).

\subsection{SIFT (Scale-Invariant Feature Transform)}

Segun (\cite{AlegreFernandez2018}) Scale invariant feature transform es un método que permite detectar puntos característicos en una imagen y luego describirlos mediante un histograma orientado de gradientes. Ademas es el reconocimiento de objetos, y especialmente cuando hay oclusión o el fondo está lleno de objetos desordenados (clutter).


